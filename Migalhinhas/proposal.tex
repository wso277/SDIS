\documentclass[a4paper]{article}

\usepackage[portuguese]{babel}
\usepackage[utf8]{inputenc}
\usepackage{indentfirst}
\usepackage{graphicx}
\usepackage{verbatim}  
\usepackage{amsmath}
\usepackage[colorinlistoftodos]{todonotes}
\usepackage{underscore}

\usepackage[T1]{fontenc}
\usepackage{listingsutf8}
\usepackage{xcolor}
\usepackage[hidelinks]{hyperref}

\usepackage{inconsolata}
\lstset{
    language=C, 
    basicstyle=\ttfamily\small,
    numberstyle=\footnotesize,
    numbers=left,
    backgroundcolor=\color{gray!10},
    frame=single,
    tabsize=2,
    rulecolor=\color{black!30},
    escapeinside={\%*}{*)},
    breaklines=true,
    breakatwhitespace=true,
    framextopmargin=2pt,
    framexbottommargin=2pt,
    extendedchars=false,
    inputencoding=utf8
}

\begin{document}

\setlength{\textwidth}{16cm}
\setlength{\textheight}{22cm}

\title{\Huge\textbf{Assignment \#2 - CrumbleUp}\linebreak\linebreak\linebreak
\Large\textbf{}\linebreak\linebreak
\includegraphics[height=6cm, width=7cm]{feup.pdf}\linebreak \linebreak
\Large{Mestrado Integrado em Engenharia Informática e Computação} \linebreak \linebreak
\Large{Sistemas Distribuídos}\linebreak
}

\author{ Fábio Filipe Jesus da Silva, ei11107@fe.up.pt \\  João Carlos Macedo Flores dos Santos, ei11126@fe.up.pt \\ João Manuel Mesquita Cardoso, ei11100@fe.up.pt \\
Wilson da Silva Oliveira, ei11085@fe.up.pt \\\linebreak\linebreak \\
 \\ Faculdade de Engenharia da Universidade do Porto \\ Rua Roberto Frias, s\/n, 4200-465 Porto, Portugal \linebreak\linebreak\linebreak
\linebreak\linebreak\vspace{1cm}}
\maketitle
\thispagestyle{empty}

%************************************************************************************************
%************************************************************************************************

\newpage

\section{Purpose of the Application}
The purpose of this application is to improve upon the previous implementation, making it suitable for a business environment.
In order to achieve this objective several features will be implemented, strengthening the application security, failure tolerancy and using a more suitable communication protocol, making use of the TCP protocol in a local network.


\section{Main Features}

\begin{itemize}
\item Security
\begin{itemize}
\item Peer autentication using encrypted passwords
\item Guaranteed chunk confidentiality using encryption whith data from the sender computer
\end{itemize}
\item Fault tolerancy
\begin{itemize}
\item Use of a write-ahead logging, so the aplication is able to recover from crash and execute integrity checks
\item Avoidance of corrupted data through cached backups of the active file
\end{itemize}
\item Communication
\begin{itemize}
\item Implementation of the TCP protocol in some cases where not all the peers need to receive the packet(s)
\item Possibility to name the communication channels making use of a dns
\end{itemize}
\item In addition to this we will implement all the previous requested enhancements
\begin{itemize}
\item Backup: After waiting a random time, only saves the chunk in the hardrive if the current replication degree is lower than the desired degree
\item Restore: Implementation of the TCP protocol to ensure that only the necessary client receives the data requested whithout fail, putting less strain on the network
\item Delete: Implement a response message to the delete request. This message will allow the delete request to be send to a recently connected client until the replication degree of this chunk reaches zero.
\item Space Reclaim: Similar to the delete protocol a list of chunks with low degree will be kept, and a request to backup a chunk will be sent to a newly connected client until the replication degree is satisfied.
\end{itemize}
\item Functionalities
\begin{itemize}
\item Backup a file in the network with a desired replication degree
\item Restore of a file previously backed up
\item Complete removal of a file previously stored in the network
\item Manage of disk space used by the application to restore chunks
\end{itemize}
\end{itemize}

\section{Target Platforms}

\begin{itemize}
\item Java standalone application for Windows/Linux/Mac
\end{itemize}

\section{Additional Services and Improvements}
If time permits us, we will develop an Android application for this project, while also allowing interoperability between all the target platforms.

\end{document}